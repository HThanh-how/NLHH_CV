\documentclass[10pt, letterpaper]{article}

% Packages:
\usepackage[
    ignoreheadfoot, 
    top=2cm, 
    bottom=2cm, 
    left=2cm, 
    right=2cm, 
    footskip=1.0cm, 
    % showframe 
]{geometry} 
\usepackage{titlesec} 
\usepackage[dvipsnames]{xcolor} 
\definecolor{primaryColor}{RGB}{0, 79, 144} 
\usepackage{enumitem} 
\usepackage{fontawesome5} 
\usepackage{amsmath} 
\usepackage[
    pdftitle={NGUYEN LUONG HUU HUY's CV},
    pdfauthor={NGUYEN LUONG HUU HUY},
    pdfcreator={LaTeX},
    colorlinks=true, % Keep colorlinks true for our custom coloring
    urlcolor=primaryColor, % Default color for URLs if not overridden
    linkcolor=primaryColor, % Default color for internal links if not overridden
    hidelinks % Removes borders and default styling, we add color back manually
]{hyperref} 
\usepackage{calc} 
\usepackage{bookmark} 
\usepackage{changepage} 
\usepackage{paracol} 
\usepackage{ifthen} 
\usepackage{needspace} 
\usepackage{iftex} 
\usepackage[utf8]{inputenc}
\usepackage{lmodern}

% Ensure that generated PDF is machine readable/ATS parsable:
\ifPDFTeX
    \input{glyphtounicode}
    \pdfgentounicode=1
\fi

% Some settings:
\AtBeginEnvironment{adjustwidth}{\partopsep0pt} 
\pagestyle{empty} 
\setcounter{secnumdepth}{0} 
\setlength{\parindent}{0pt} 
\setlength{\topskip}{0pt} 
\setlength{\columnsep}{0cm} 

\titleformat{\section}{\needspace{4\baselineskip}\bfseries\large}{}{0pt}{}[\vspace{1pt}\titlerule]

\titlespacing{\section}{
    -1pt 
}{
    0.4cm 
}{
    0.2cm
} 

\renewcommand\labelitemi{$\circ$} 
\newenvironment{highlights}{
    \begin{itemize}[
        topsep=0.10cm,
        parsep=0.10cm,
        partopsep=0pt,
        itemsep=0.05cm,
        leftmargin=0.4cm + 10pt 
    ]
}{
    \end{itemize}
} 

\setlist[itemize,2]{ 
    label=\textbullet, 
    leftmargin=1em,   
    topsep=0.05cm,
    partopsep=0pt,
    parsep=0.05cm,
    itemsep=0.02cm    
}
\setlist[itemize,3]{ 
    label=\textendash,
    leftmargin=1em,
    topsep=0.05cm,
    partopsep=0pt,
    parsep=0.05cm,
    itemsep=0.02cm
}

\newenvironment{onecolentry}{
    \begin{adjustwidth}{
        0.2cm 
    }{
        0.2cm
    }
}{
    \end{adjustwidth}
} 

\newenvironment{twocolentry}[2][]{ 
    \onecolentry
    \def\secondColumnText{#2}
    \setcolumnwidth{\fill, 5.0cm} 
    \begin{paracol}{2}
    \ifblank{#1}{}{\textbf{#1}\\}% 
}{
    \switchcolumn \raggedleft \textit{\secondColumnText} 
    \end{paracol}
    \endonecolentry
} 

\newenvironment{header}{
    \setlength{\topsep}{0pt}\par\kern\topsep\centering\linespread{1.1} 
}{
    \par\kern\topsep
} 

% Store original hyperref's \href command
\let\RealHref\href

% Redefine \href for general use (if any in body, currently none)
% This will color the text primaryColor
\renewcommand{\href}[2]{\RealHref{#1}{\textcolor{primaryColor}{#2}}}

% New command for header links, using black text
\newcommand{\headerlink}[2]{\RealHref{#1}{\textcolor{black}{#2}}}


\begin{document}
\pagestyle{empty} 

\begin{header}
    \textbf{\fontsize{24pt}{24pt}\selectfont NGUYEN LUONG HUU HUY}

    \vspace{0.15cm} 

    \normalsize
    Ho Chi Minh City \\[0.05cm] % Small break after city
    \mbox{\color{black}\footnotesize\faPhone*}\hspace*{0.13cm}0946790659
    \kern 0.3cm 
    \mbox{\color{black}\footnotesize\faEnvelope[regular]}\hspace*{0.13cm}\headerlink{mailto:huy.nguyenhuuhuybk@hcmut.edu.vn}{huy.nguyenhuuhuybk@hcmut.edu.vn} \\[0.1cm] % Line break and small space
    \mbox{\color{black}\footnotesize\faLinkedin}\hspace*{0.13cm}\headerlink{https://www.linkedin.com/in/huy-nguy\%E1\%BB\%85n-l\%C6\%B0\%C6\%A1ng-h\%E1\%BB\%AFu-8b322b310/}{linkedin.com/in/huy-nguyen-luong-huu} % Escaped %
    \kern 0.3cm 
    \mbox{\color{black}\footnotesize\faGithub}\hspace*{0.13cm}\headerlink{https://github.com/HuyBK0408}{github.com/HuyBK0408}
\end{header}

\vspace{0.3cm}

\section{OBJECTIVE}
\begin{onecolentry}
    A highly motivated and aspiring Computer Engineering graduate seeking a challenging entry-level position to apply foundational knowledge in embedded systems development, circuit design, and hardware engineering. Eager to contribute to innovative projects, particularly those in IoT and automation, while learning from experienced professionals and growing within a dynamic company that values continuous improvement and collaborative problem-solving.
\end{onecolentry}

\section{EDUCATION}
\begin{twocolentry}[HO CHI MINH CITY UNIVERSITY OF TECHNOLOGY]{AUGUST 2019 -- MAY 2025}
    \textit{Major: Computer Engineering} \\
    GPA: 6.0/10 (Equivalent to 2.4/4.0 scale) \\[4pt]
    \textbf{Relevant Coursework:} Data Structures & Algorithms, Microprocessors & Microcontrollers, Embedded Systems Design, Computer Architecture, Operating Systems, Circuit Theory, Digital Logic Design, Communication Systems. \\
    \textbf{Key Academic Projects/Activities:}
    \begin{highlights}
        \item Designed and implemented an 8-bit Arithmetic Logic Unit (ALU) using VHDL and simulated in ModelSim.
        \item Developed a line-following robot programmed in C for a university robotics competition, achieving second place.
        \item Senior Design Project: IoT-based Smart Environmental Monitoring System for urban areas - Completed.
    \end{highlights}
    \textbf{Final Year Project:} Title: "Development of a Low-Power Wide-Area Network (LPWAN) Node for Agricultural Monitoring" - Completed, focusing on LoRa communication and sensor integration.
\end{twocolentry}
\vspace{0.2cm}
\begin{twocolentry}[LE KHIET HIGH SCHOOL FOR THE GIFTED]{SEPTEMBER 2016 -- JUNE 2019}
    \textit{Quang Ngai, Vietnam} \\
    Specialized in Mathematics
\end{twocolentry}

\section{HONORS \& AWARDS}
\begin{onecolentry}
    \begin{highlights}
        \item Provincial-level Consolation Prize, Excellent Student Competition in Mathematics (Grade 9) \hfill \textit{2016}
        \item Third Prize, City-level Excellent Student Competition in Mathematics (Grade 8) \hfill \textit{2015}
        \item Provincial-level Consolation Prize, Mathematics & English Olympiad (Primary School) \hfill \textit{2012}
    \end{highlights}
\end{onecolentry}

\section{CERTIFICATIONS}
\begin{onecolentry}
    \begin{highlights}
        \item IELTS Certificate (Overall Band Score 6.0) \hfill \textit{Issued 2022}
        \item Certificate of Completion - "Embedded Systems Essentials with Arm" - Arm Education \hfill \textit{2023}
    \end{highlights}
\end{onecolentry}

\section{SKILLS}
\begin{onecolentry}
    \begin{highlights}
        \item \textbf{Languages:}
            \begin{itemize}
                \item English: Proficient (IELTS 6.0)
            \end{itemize}
        \item \textbf{Programming Languages:}
            \begin{itemize}
                \item C/C++: Knowing HTML, CSS, Proficient (Extensive use in microcontroller programming (STM32, Arduino), algorithm implementation, embedded firmware development for academic and personal projects).
                \item Assembly Language: Basic (Familiar with ARM Cortex-M and MIPS assembly for specific coursework on computer architecture and low-level debugging).
            \end{itemize}
        \item \textbf{Embedded Systems &Hardware:}
            \begin{itemize}
                \item Microcontrollers/Processors: Arduino (AVR family), STM32 (F1, F4, L4 series), Raspberry Pi (3/4), ESP32/ESP8266.
                \item Development Tools: STM32CubeIDE, Keil MDK, PlatformIO, Proteus (Circuit Design & Simulation), GCC toolchain, GDB, Eagle/KiCad (for basic PCB design and review).
                \item Hardware Interfacing: Sensors (temperature, humidity, ultrasonic, IR, accelerometer, gyroscope), Actuators (motors, servos, relays), Communication Protocols & Interfaces (SPI, I2C, UART, ADC, DAC), Wireless Modules (Bluetooth Low Energy, Wi-Fi, LoRa).
                \item Concepts: Real-Time Operating Systems (FreeRTOS), Digital Signal Processing (DSP fundamentals), Low-Power Design Techniques, Debugging (JTAG/SWD interfaces, logic analyzers).
            \end{itemize}
        \item \textbf{Software &Tools:}
            \begin{itemize}
                \item Version Control: Git, GitHub (Proficient with branching, merging, and pull requests).
                \item Operating Systems: Windows, Linux (Ubuntu distribution primarily, comfortable with command line).
                \item IDEs: VS Code, Eclipse, CLion.
                \item Others: MATLAB (for algorithm simulation and control system design), LabVIEW (basic exposure for data acquisition and instrument control).
            \end{itemize}
       \item \textbf{Methodologies:} Familiar with Agile (Scrum) concepts from team projects. Experience with Waterfall model application in academic coursework.
       \item \textbf{Soft Skills:} Problem-Solving, Analytical Thinking, Teamwork & Collaboration, Effective Communication (written and verbal), Adaptability, Continuous Learning, Attention to Detail, Time Management.
    \end{highlights}
\end{onecolentry}
\vspace{0.1cm} 

\section{PROJECTS}
\vspace{0.1cm}
\begin{twocolentry}[Project: Blockchain-based Peer-to-Peer Energy Trading Platform]{MARCH 2023 -- JUNE 2023}
    \textit{Role: Developer (University Course Project)}
\end{twocolentry}
\begin{onecolentry}
    \begin{highlights}
        \item \textbf{Description:} Conceptualized and developed a prototype for a decentralized energy trading platform, enabling simulated peer-to-peer energy transactions using embedded devices (Arduino) and foundational blockchain principles. Focused on the hardware-software interface for energy monitoring and secure data communication.
        \item \textbf{Key Responsibilities:}
            \begin{itemize}
                \item Designed and programmed firmware in C/C++ for Arduino-based embedded nodes to simulate energy consumption/generation, and communicate encrypted data packets.
                \item Implemented MQTT protocol for transmitting simulated energy data to a Raspberry Pi acting as a local broker and data aggregator.
                \item Utilized Proteus for comprehensive circuit design, simulation, and validation of embedded node hardware before physical prototyping.
                \item Collaborated within a team of 5 to define system architecture and integrate modules, leading the embedded systems development track.
                \item Documented embedded system functionalities, communication interfaces, and test procedures.
                \item Researched and implemented a lightweight data hashing algorithm (SHA-256 snippets) for ensuring data integrity.
            \end{itemize}
        \item \textbf{Key Achievements/Features:}
            \begin{itemize}
                \item Successfully demonstrated real-time (simulated) data logging from multiple energy nodes.
                \item Developed a functional embedded device prototype capable of measuring mock energy data and transmitting it securely over a local network.
                \item Gained practical experience in embedded C programming, secure communication protocols, and collaborative development using Git.
            \end{itemize}
        \item \textbf{Technologies Used:}
            \begin{itemize}
                \item Hardware: Arduino Uno R3, ESP8266 (for Wi-Fi), Current Sensors (ACS712), Voltage Sensors, Raspberry Pi 3B+ (as MQTT Broker).
                \item Software & Tools: C/C++ (Arduino IDE, PlatformIO), Proteus ISIS/ARES, Git, GitHub.
                \item Concepts: Blockchain (fundamental principles), Peer-to-Peer Networking, MQTT, Data Hashing.
            \end{itemize}
    \end{highlights}
\end{onecolentry}
\vspace{0.3cm}

\begin{twocolentry}[Project: Automated Plant Watering System with IoT Monitoring]{SEPTEMBER 2023 -- DECEMBER 2023} 
    \textit{Role: Lead Developer (Personal Hobby Project)}
\end{twocolentry}
\begin{onecolentry}
    \begin{highlights}
        \item \textbf{Description:} Designed and built an automated plant watering system using an ESP32 microcontroller, soil moisture sensors, and a relay-controlled water pump. The system monitors environmental conditions and provides status updates.
        \item \textbf{Key Responsibilities:}
            \begin{itemize}
                \item Developed firmware in C/C++ using the Arduino framework for ESP32 to read data from soil moisture sensors, temperature/humidity sensors (DHT11).
                \item Implemented control logic for a DC water pump via a relay module, based on configurable soil moisture thresholds.
                \item Enabled system status indication via LEDs and serial monitor output for diagnostics.
                \item Designed and assembled the hardware components, including sensor integration and power management considerations.
            \end{itemize}
        \item \textbf{Key Achievements/Features:}
            \begin{itemize}
                \item Achieved fully autonomous watering cycles, maintaining optimal soil moisture levels for indoor plants.
                \item Successfully deployed a stand-alone embedded system for environmental control.
                \item Enhanced understanding of low-power modes on ESP32 for potential battery operation.
            \end{itemize}
        \item \textbf{Technologies Used:}
            \begin{itemize}
                \item Hardware: ESP32 Dev Kit, Soil Moisture Sensors (Capacitive), DHT11 Sensor, 5V Relay Module, Mini DC Water Pump.
                \item Software & Tools: C/C++ (Arduino IDE with ESP32 Core), Fritzing (for circuit diagram).
            \end{itemize}
    \end{highlights}
\end{onecolentry}
\vspace{0.3cm}



\section{INTERESTS}
\begin{onecolentry}
    Passionate about team sports like football, which enhances teamwork and strategic thinking. I also engage in strategy-based simulation software, fostering problem-solving skills. Reading novels, particularly in science fiction and technology ethics genres, broadens my perspectives. Additionally, I have a keen interest in tinkering with hobby electronics and building IoT devices, following advancements in embedded AI and sustainable green technologies, and contributing to open-source embedded firmware.
\end{onecolentry}

\end{document}